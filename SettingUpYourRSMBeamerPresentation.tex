\documentclass[11pt]{scrartcl}
\usepackage{hyperref}
\usepackage{verbatim}

\begin{document}
\bibliographystyle{plainnat}
\title{Setting up your RSM beamer presentation}
\maketitle	

\noindent The goal of this guide is to help you set up a presentation in LaTeX using the RSM beamer template, by showing you which steps you have to take and where you can find the required files. First, you will create a central directory on your computer. Subsequently, you will download and save the template file in that directory. Finally, you will tell MiKTeX how to find the templates when you invoke the rsmbeamer class.

\section*{Step 1.}
Create a directory in which you can store the RSM beamer template. A safe way to set this up is by creating the following directory structure: \vspace{1em}

\noindent\path{C:\ Latexfiles\ tex\ latex\ rsmbeamer} \vspace{1em}

\noindent Here it is assumed that C:\ is your local hard drive name.

\section*{Step 2.}
Download the RSM beamer presentation temmplate here: \vspace{1em}

\noindent \url{http://www.rsm.nl/brandtoolbox/downloads/presentation-templates/} \vspace{1em}

\noindent Save all the files within rsmbeamer\_v2\_0 in the rsmbeamer directory that you have created in step 1.

\section*{Step 3.}
Show MiKTeX where to find your rsmbeamer file: \vspace{1em}

\noindent Click start $>$ Programs $>$ MiKTeX 2.9 $>$ Maintenance $>$ Settings. Now, click on the Roots tab and click 'Add'. Browse to the path \path{C:\ Latexfiles} and click OK. Go to the tab 'General' and click 'Refresh FNDB'. Click OK.

\section*{Bonus}
To start you off with a presentation, open a new file in TeXstudio, or WinEdt, etc.; include and run the following code: \vspace{1em}

\noindent \begin{verbatim}
\documentclass[research,t,11pt]{RSMBeamer}

\title[Title in bottom of slides]{Title on first page}
\author{Your Name}
\institute{RSM}
\date{May 23rd 2014}

\begin{document}

\begin{frame}
\maketitle
\end{frame}

\section{Introduction}
\begin{frame}
\frametitle{Topic}

\end{frame}

\begin{frame}
\frametitle{Summary of Findings}

\end{frame}

\section{Data}
\begin{frame}
\frametitle{Data sources}

\end{frame}

\section{Approach}
\begin{frame}
\frametitle{}

\end{frame}

\section{Results}
\begin{frame}
\frametitle{}

\end{frame}

\section{Conclusion}
\begin{frame}
\frametitle{Conclusion}

\end{frame}

\end{document}

\end{verbatim}

\section{Sources}
\noindent \url{https://intranet.rsm.eur.nl} \vspace{1em}


\end{document} 
